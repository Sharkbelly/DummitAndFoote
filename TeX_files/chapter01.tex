\chapter{Introduction to Groups}
\section{Basic Axioms and Examples}
Let $G$ be a group.
	\begin{enumerate}
		\item Determine which of the following binary operations are associative:
		\begin{enumerate}
			\item the operation $\star$ on $\integers$ defined by $a\star b = a-b$
			\item the operation $\star$ on $\reals$ defined by $a\star b =a+b+ab$
			\item the operation $\star$ on $\rationals$ defined by $a\star b = \dfrac{a+b}{5}$
			\item the operation $\star$ on $\integers\times\integers$ defined by $(a,b)\star(c,d) = (ad+bc,bd)$
			\item the operation $\star$ on $\rationals  - \left\{0\right\}$ defined by $a\star b =\dfrac{a}{b}$
		\end{enumerate}
		\item Decide which of the binary operations in the preceding exercise are commutative.
		\item Prove that addition of the residue classes in $\intgp{n}$ is associative.  Assume it is well defined.
		\item Prove that multiplication of residue classes in $\intgp{n}$ is associative.  Assume it is well defined.
		\item Prove for all $n>1$ that $\intgp{n}$ is not a group under multiplication of residue classes.
		\item Determine which of the following sets are groups under addition:
		\begin{enumerate}
			\item Rational numbers in lowest terms whose denominators are odd.
			\item Rational numbers in lowest terms whose denominators are even.
			\item Rational numbers of absolute value less than $1$.
			\item Rational numbers of absolute value greater than or equal to $1$ and including $0$.
			\item Rational numbers with denominators equal to $1$ or $2$.
			\item Rational numbers with denominators equal to $1$, $2$, or $3$.
		\end{enumerate}
		\item Let $G=\left\{x\in\reals|0\leq x<1 \right\}$ and for $x,y\in G$, let $x\star y$ be the fractional part of $x+y$.  Prove that $\star$ is a well defined binary operation on $G$ and $G$ is an abelian group under $\star$ (called the \emph{real numbers $\pmod 1$}).
		\item Let $G = \left\{z=in \complexs | z^n = 1 \text{ for some }n\in\integers\plus \right\}$.
		\begin{enumerate}
			\item Prove that $G$ is a group under multiplication (called the \emph{group of roots of unity in $\complexs$}).
			\item Prove that $G$ is not a group under addition.
		\end{enumerate}
		\item Let $G = \left\{ a+b\sqrt{2}\in\reals|a,b\in\rationals \right\}$.
		\begin{enumerate}
			\item Prove that $G$ is a group under addition.
			\item Prove that the nonzero elements of $G$ are a group under multiplication.
		\end{enumerate}
		\item Prove that a finite group is abelian if and only if its group table is a syymmetric matrix.
		\item Find the orders of each element of the additive group $\intgp{12}$.
		\item Find the orders of the following elements of the multiplicative group $(\intgp{12})^\times$: $\res{1},\res{-1},\res{5},\res{7},\res{-7},\res{13}$.
		\item Find the orders of the following elements of the additive group $\intgp{36}$: $\res{1},\res{2},\res{6},\res{9},\res{10},\res{12},\res{-1},\res{-10},\res{-18}$.
		\item Find the orders of the following elements of the additive group $(\intgp{36})^\times$: $\res{1},\res{-1},\res{5},\res{13},\res{-13},\res{17}$.
		\item Prove that $(a_1a_2...a_n)^{-1} = a_n\inv a_{n-1}\inv ... a_1\inv$ for all $a_i\in G$.
		\item Let $x\in G$.  Prove that $x^2=1$ if and only if $|x|$ is either $1$ or $2$.
		\item Let $x\in G$.  Prove that if $|x| = n$ for some positive integer $n$, then $x\inv = x^{n-1}$.
		\item Let $x,y\in G$.  Prove that $xy = yx$ if and only if $y\inv x y = x$ if and only if $x\inv y\inv xy = 1$.
		\item Let $x\in G$ and let $a,b \in\integers\plus$.
		\begin{enumerate}
			\item Prove that $x^{a+b} = x^ax^b$ and $(x^a)^b = x^{ab}$.
			\item Prove that $(x^a)\inv = x^{-a}$.
			\item Establish part (a) for arbitrary integers $a$ and $b$.
		\end{enumerate}
		\item For $x\in G$, show that $x$ and $x\inv$ have the same order.
		\item Let $G$ be a finite group and let $x\in G$ be of order $n$.  Prove that if $n$ is odd, then $x = (x^2)^k$ for some $k$.
		\item If $x$ and $g$ are elements of $G$, prove that $|x|=|g\inv xg|$.  Deduce that $|ab| = |ba|$ for all $a,b\in G$.
		\item Suppose $x\in G$ and $|x| = n<\infty$.  If $n=st$ for some positive integers $s$ and $t$, prove that $|x^s| = t$.
		\item If $a$ and $b$ are commuting elements of $G$, prove that $(ab)^n = a^nb^n$ for all $n\in \integers$.
		\item Prove that if $x^2 = 1$ for all $x\in G$, then $G$ is abelian.
		\item Assume $H$ is a nonempty subset of $(G,\star)$ which is closed under $\star$ and under inverses.  Prove that $H$ is a group under $\star$ restricted to $H$ ($H$ is a \emph{subgroup of $G$}).
		\item Prove that if $x$ is an element of the group $G$, then $\left\{ x^n|n\in\integers \right\}$ is a subgroup of $G$ (called the \emph{cyclic subgroup of $G$ generated by $x$}).
		\item Let $(A,\star)$ and $(B,\diamond)$ be groups and let $A\times B$ be their direct procuct.  Verify all the group axioms for $A\times B$.
		\begin{enumerate}
			\item Prove that the associative law holds: for all $(a_i,b_i)\in A\times B, i = 1,2,3$, \\$(a_1,b_1)[(a_2,b_2)(a_3,b_3)] = [(a_1,b_1)(a_2,b_2)](a_3,b_3))$,
			\item Prove that $(1,1)$ is the identity of $A\times B$, and
			\item Prove that the inverse $(a,b)$ is $(a\inv,b\inv)$.
		\end{enumerate}
		\item Prove that $A\times B$ is an abelian group if and only if $A$ and $B$ are.
		\item Prove that the elements $(a,1)$ and $(1,b)$ of $A\times B$ commuted and the order of $(a,b)$ is the least common multiple of $|a|$ and $|b|$.
		\item Prove that any finite group $G$ of even order contains an element of order $2$.
		\item If $x$ is an element of finite order $n$ in $G$, prove that the elements $1$, $x$, $x^2$, ..., $x^{n-1}$ are all distinct.  Deduce that $|x|<G$.
		\item Let $x$ be an element of finite order $n$ in $G$.
		\begin{enumerate}
			\item Prove that if $n$ is odd then $x^i \neq x^{-i}$ for all $i =  1,2,...,n-1$.
			\item Prove that if $n = 2k$ and $1\leq i<n$ then $x^i = x^{-i}$ if and only if $i = k$.
		\end{enumerate}
		\item If $x$ is an element of infinite order in $G$, prove that the elements $x^n$, $n\in\integers$ are all distinct.
		\item If $x$ is an element of finite order in $G$, use the Division Algorithm to show that any integral power of $x$ equals one of the elements in the set of $\left\{ 1,x,x^2,...,x^{n-1} \right\}$.
		\item Assume $G = \left\{ 1,a,b,c \right\}$ is a group of order $4$.  Assume also that $G$ has no element of order $4$.  Use the cancellation laws to show that there is a unique group table for $G$.  Deduce that $G$ is abelian.
	\end{enumerate}
