\chapter{Introduction to Groups}
\section{Basic Axioms and Examples}
Let $G$ be a group.
	\begin{enumerate}
		\item Determine which of the following binary operations are associative:
			\begin{enumerate}
				\item the operation $\star$ on $\integers$ defined by $a\star b = a-b$
				\item the operation $\star$ on $\reals$ defined by $a\star b =a+b+ab$
				\item the operation $\star$ on $\rationals$ defined by $a\star b = \dfrac{a+b}{5}$
				\item the operation $\star$ on $\integers\times\integers$ defined by $(a,b)\star(c,d) = (ad+bc,bd)$
				\item the operation $\star$ on $\rationals  - \left\{0\right\}$ defined by $a\star b =\dfrac{a}{b}$
			\end{enumerate}
			\begin{sln}
			\begin{enumerate}
				\item This is not an associative binary operation.
				\begin{proof}
					Consider the case where $a=1, b=2, c=2$.  Then
					$$a\star(b\star c) = 1\star (2\star 2) = 1\star (2-2) = 1 \star 0 = 1-0 = 1$$
					Conversely,
					$$(a\star b)\star c = (1- 2)\star 2) = -1\star 2  = -1 - 2 = -3$$
				\end{proof}
				\item This is an associative binary operation.
					\begin{proof}
						We will show that 
						\[
							a\star(b\star c) = (a\star b)\star c 
						\]
						By definition,
						\begin{align*}
							a\star(b\star c) &= a\star (b+c+bc) \\
							&= a + (b + c + bc)+a(b + c + bc)\\
							&= a+b+c+bc+ab+ac+abc
						\end{align*}
						
						Also,
						\begin{align*}
						(a\star b)\star c &= (a + b + ab) \star c\\
						&= (a+b+ab) + c + (a+b+ab)c\\
						&= a+b+c+ab+ac+bc+abc\\
						\end{align*}
						Since addition over the integers is associative, these quantities are equivalent.
					\end{proof}
				\item This is not an associative binary operation.
					\begin{proof}
						Observe that 
						\[a\star(b\star c) = \dfrac{5a+b+c}{25}\]
						and
						\[(a\star b)\star c = \dfrac{a+b+5c}{25}\]
						To see that these quantities are not equivalent for all $a, b,$ and $c$, let $a=1,b=2,c=3$:
						\[a\star(b\star c) = \dfrac{5+ 2 + 3}{25} = \dfrac{10}{25}\]
						and
						\[(a\star b)\star c = \dfrac{1 + 2 + 15}{25} = \dfrac{18}{25}\]
					\end{proof}
				\item This is an associative binary operation.
					\begin{proof}
						We will show that 
						\[
						a\star(b\star c) = (a\star b)\star c 
						\]
						Observe that 
						\begin{align*}
							(a,b)\star[(c,d)\star(f,g)] &= (a,b)\star(cg+df,dg)\\
							&= (adg+bcg+bdf,bdg)
						\end{align*}
						and
						\begin{align*}
						[(a,b)\star(c,d)]\star(f,g) &= (ad+bc,bd)\star(f,g)\\
						&= (adg+bcg+bdf,bdg)
						\end{align*}
					\end{proof}
					\item This is not an associative binary operation.
					\begin{proof}
						Observe that 
						\[
							a\star (b \star c) = a\star \dfrac{b}{c} = \dfrac{a}{\frac{b}{c}}= \dfrac{ac}{b}
						\]
						while
						\[
							(a\star b) \star c = \dfrac{a}{b}\star c= \dfrac{\frac{a}{b}}{c}
							= \dfrac{a}{bc}
						\]
					\end{proof}
			\end{enumerate}
		\end{sln}
		\item Decide which of the binary operations in the preceding exercise are commutative.
			\begin{sln}
				\begin{enumerate}
					\item This is not a commutative binary operation.
						\begin{proof}
							Observe that $a\star b = a-b$ while $b\star a = b-a$.  This are clearly not equivalent if we suppose $a = 1$ and $b = 0$.  The first expression evaluates to $1$ while the second evaluates to $-1$.
						\end{proof}
					\item This is a commutative binary operation.
						\begin{proof}
							Observe that $a\star b = a+b+ab$ and $b\star a = b+a+ba$.  Since both multiplication and addition are commutative, these expressions are equivalent.
						\end{proof}
					\item This is a commutative binary operation.
						\begin{proof}
							Observe that \[a\star b = \dfrac{a+b}{5} = \dfrac{b+a}{5} = b \star a\]
						\end{proof}
					\item This is a commutative binary operation.
						\begin{proof}
							Observe that \[(a,b)\star (c,d) = (ad+bc, bd) = (cb+da, db) = (c,d)\star (a,b)\]
						\end{proof}
					\item This is not a commutative binary operation.
						\begin{proof}
							Observe that \[a\star b = \dfrac{a}{b}\]
							and \[b\star a = \dfrac{b}{a}\]
							For $a = 1, b=2$, $a\star b = \frac{1}{2}$ and $b\star a = 2$.
						\end{proof}
				\end{enumerate}
			\end{sln}
		\item Prove that addition of the residue classes in $\intgp{n}$ is associative.  Assume it is well defined.
			\begin{proof}
				Let $\overline{a}, \overline{b}, \overline{c} \in \intgp{n}$.  Then by definition,
				\begin{align*}
					\res{a} + (\res{b} + \res{c}) & = \res{a} + \res{b+c}\\
					&=\res{a+(b+c)}\\
					&=\res{(a+b)+c}\\
					&=\res{a+b} + \res{c}\\
					&=(\res{a}+\res{b})+\res{c}
				\end{align*}
				We see that addition of resedue classes is associative.
			\end{proof}
		\item Prove that multiplication of residue classes in $\intgp{n}$ is associative.  Assume it is well defined.
			\begin{proof}
				Let $\overline{a}, \overline{b}, \overline{c} \in \intgp{n}$.  Then by definition,
				\begin{align*}
				\res{a} \cdot (\res{b} \cdot \res{c}) & = \res{a} \cdot \res{b\cdot c}\\
				&=\res{a\cdot(b\cdot c)}\\
				&=\res{(a\cdot b)\cdot c}\\
				&=\res{a\cdot b} \cdot \res{c}\\
				&=(\res{a}\cdot\res{b})\cdot\res{c}
				\end{align*}
				We see that multiplication of resedue classes is associative.
			\end{proof}
		\item Prove for all $n>1$ that $\intgp{n}$ is not a group under multiplication of residue classes.
			\begin{proof}
				Let $\res{a} = \res{0}$.  Then for any $\res{b}\in\intgp{n}$, we have $\res{a}\res{b} = 0\neq \res{1}$.  Thus, $\res{0}$ has no inverse in the set under multiplication.
			\end{proof}
		\item Determine which of the following sets are groups under addition:
		\begin{enumerate}
			\item $G=$ Rational numbers in lowest terms whose denominators are odd.
			\item $G=$ Rational numbers in lowest terms whose denominators are even.
			\item $G=$ Rational numbers of absolute value less than $1$.
			\item $G=$ Rational numbers of absolute value greater than or equal to $1$ and including $0$.
			\item $G=$ Rational numbers with denominators equal to $1$ or $2$.
			\item $G=$ Rational numbers with denominators equal to $1$, $2$, or $3$.
		\end{enumerate}
		\begin{sln}
			\begin{enumerate}
				\item Not sure.
				
				\item This is not a group.
					\begin{proof}
						We will show the group is not closed under the operation.
						\[\frac{1}{2} + \frac{1}{2} = 1\notin G\]
					\end{proof}
				
				\item This is not a group.
					\begin{proof}
						We will show the group is not closed under the operation.
						\[\frac{3}{4} + \frac{3}{4} = 1\notin G\]
					\end{proof}
				
				\item This is not a group.
					\begin{proof}
						We will show the group is not closed under the operation.  Consider $a=3.25$ and $b=-3$.
						\[a+b = 3.25 - 3 = 0.25\notin G\]
					\end{proof}
				\item This is a group.
					\begin{proof}
						We take it as understood that addition of rational numbers is a well defined binary operation.  We will show first that $G$ is closed under addition.\\
						Suppose $a,b\in G$.  Then one of the following must be true:
						\begin{enumerate}
							\item $a = \dfrac{m}{1}, b = \dfrac{n}{1}$
							\item $a = \dfrac{m}{1}, b = \dfrac{n}{2}$
							\item $a = \dfrac{m}{2}, b = \dfrac{n}{2}$
						\end{enumerate}
						In the first case, their sum is an integer, so its denominator is $1$.  \\In the second case, \[a+b = \dfrac{m}{1} +\dfrac{n}{2} = \dfrac{2m}{2} +\dfrac{n}{2} = \dfrac{2m+n}{2}\]
						In the event that the $2m+n$ is even, a factor of $2$ can cancel, leaving $1$ as the denominator; therefore, the sum $a+b$ is in $G$.  If no cancellation can be performed, the sum is still in $G$.\\
						Finally, \[a+b = \dfrac{m}{2}+\dfrac{n}{2} = \dfrac{m+n}{2}\]  By similar reasoning to the preceding case, this sum is in $G$.\\
						The identity element of this group is $0$ (written as $\frac{0}{1}$).\\
						If $a\in G$, then $-a$ is the additive inverse (and it is clearly also in $G$).
					\end{proof}
				\item This is not a group.
					\begin{proof}
						We will show the group is not closed under the operation.
						\[\frac{1}{3} + \frac{1}{2} = \frac{5}{6}\notin G\]
					\end{proof}
			\end{enumerate}
		\end{sln}
		\item Let $G=\left\{x\in\reals|0\leq x<1 \right\}$ and for $x,y\in G$, let $x\star y$ be the fractional part of $x+y$.  Prove that $\star$ is a well defined binary operation on $G$ and $G$ is an abelian group under $\star$ (called the \emph{real numbers $\pmod 1$}).
		\begin{proof}
			Let $x,y\in G$.  Observe that since $x,y<1$, either $\lfloor x+y \rfloor = 0$ or $\lfloor x+y \rfloor =1$.  If $x+y<1$, then $x\star y = x+y \in G$.  Otherwise, since $x,y\in G$, $x+y<2$, so $\lfloor x+y \rfloor = 1$, and $x\star y = x+y - 1 <1 \in G$.\\
			The additive identity is $0$, since $0\in G$ and $x+0 = x\in G$ for all $x\in G$.\\
			Let $x\in G$.  Either $x=0$ or $0<x<1$.  If $x=0$, then $x^{-1} = 0\in G$ because $0+0=0$.  If $x\neq 0$, then $x^{-1} = 1-x$.  To see this, first note that $x-1\in G$  This follows from $0<x<1$.  Now observe $x\star (x-1) = x+(1-x) - \lfloor x+(1-x)\rfloor = 1-\lfloor 1\rfloor = 1-1 = 0$.
			Associativity can be shown as follows:
			\begin{align*}
				a\star (b \star c) &= a+b+c-\lfloor b + c \rfloor - \lfloor a+b+c - \lfloor b+c\rfloor\rfloor\\
				& a+b+c-\lfloor a+b+c\rfloor\\
				& a+b+c-\lfloor a + b \rfloor -\lfloor a+b+c-\lfloor a+b\rfloor\rfloor\\
				& = (a\star b)\star c
			\end{align*}  
			Commutativity holds over all integers.  Observe that $x\star y = x+y - \lfloor x+y\rfloor = y+x - \lfloor y+x\rfloor = y\star x$.\\
			It is proven that this is an abelian group.
			
		\end{proof}
		\item Let $G = \left\{z\in \complexs | z^n = 1 \text{ for some }n\in\integers\plus \right\}$.
		\begin{enumerate}
			\item Prove that $G$ is a group under multiplication (called the \emph{group of roots of unity in $\complexs$}).
			\begin{proof}
				Let $z_1, z_2 \in G$.  Then there exist $n_1,n_2\in\integers\plus$ such that $z_1^{n_1} = z_2^{n_2} = 1$.  Let $n = n_1n_2$.  Then $(z_1\cdot z_2)^n = z_1^n z_2^n = z_1^{n_1n_2}z_2^{n_1n_2} = (z_1^{n_1})^{n_2}(z_2^{n_2})^{n_1} = 1^{n_2}1^{n_1}=1$, so the set is closed under the operation of multiplication.\\
				We will show that $e=1$ is the identity.  First, we must show that $1\in G$ which follows trivially because $1^n = 1$ for any $n\in \integers\plus$.  Then since for any $z\in \complexs$, $z\cdot 1 = 1\cdot z = z$, we see that $1$ is the identity element.\\
				Let $z_1=a+bi, z_2=c+di, z_3=f+gi \in G$.  Consider 
				\begin{align*}
					z_1(z_2z_3) &= (a+bi)[(c+di)(f+gi)]\\
					& (a+bi)[(cf-dg)+(df+cg)i]\\
					& acf-adg -bdf-bcd+(bcf-bdg+adf+acg)i
				\end{align*}
				Similarly
				\begin{align*}
					(z_1z_2)z_3 &= [(a+bi)(c+di)](f+gi)\\
					&=[(ac-bd)+(bc+ad)i](f+gi)\\
					&=acf-bdf-bcg-adg+(fbc+fad+gac-gbd)i
				\end{align*}
				Down to rearrangement, these are equivalent, so the operation is associative.\\
				Letting $z\in G$, the inverse of $z$ is $\frac{1}{z}$.  To see this, note that $z\cdot\frac{1}{z} = \frac{z}{z} = \frac{1}{z}z = 1$.  For such $z$, there exists $n\in \integers\plus$, so consider $\left(\frac{1}{z}\right)^n = \frac{1^n}{z^n} = \frac{1}{1} = 1$.  Thus the inverse of $z\in G$ is also in $G$.
			\end{proof}
			\item Prove that $G$ is not a group under addition.
			\begin{proof}
				Note: $1\in G$, but $1+1 = 2$ and for no positive power of $n$ is $2^n = 1$.  So the group is not closed under addition.
			\end{proof}
		\end{enumerate}
		\item Let $G = \left\{ a+b\sqrt{2}\in\reals|a,b\in\rationals \right\}$.
		\begin{enumerate}
			\item Prove that $G$ is a group under addition.
			\begin{proof}
			\end{proof}
			\item Prove that the nonzero elements of $G$ are a group under multiplication.
			\begin{proof}
			\end{proof}
		\end{enumerate}
		\item Prove that a finite group is abelian if and only if its group table is a symmetric matrix.
		\item Find the orders of each element of the additive group $\intgp{12}$.
		\item Find the orders of the following elements of the multiplicative group $(\intgp{12})^\times$: $\res{1},\res{-1},\res{5},\res{7},\res{-7},\res{13}$.
		\item Find the orders of the following elements of the additive group $\intgp{36}$: $\res{1},\res{2},\res{6},\res{9},\res{10},\res{12},\res{-1},\res{-10},\res{-18}$.
		\item Find the orders of the following elements of the additive group $(\intgp{36})^\times$: $\res{1},\res{-1},\res{5},\res{13},\res{-13},\res{17}$.
		\item Prove that $(a_1a_2...a_n)^{-1} = a_n\inv a_{n-1}\inv ... a_1\inv$ for all $a_i\in G$.
		\item Let $x\in G$.  Prove that $x^2=1$ if and only if $|x|$ is either $1$ or $2$.
		\item Let $x\in G$.  Prove that if $|x| = n$ for some positive integer $n$, then $x\inv = x^{n-1}$.
		\item Let $x,y\in G$.  Prove that $xy = yx$ if and only if $y\inv x y = x$ if and only if $x\inv y\inv xy = 1$.
		\item Let $x\in G$ and let $a,b \in\integers\plus$.
		\begin{enumerate}
			\item Prove that $x^{a+b} = x^ax^b$ and $(x^a)^b = x^{ab}$.
			\item Prove that $(x^a)\inv = x^{-a}$.
			\item Establish part (a) for arbitrary integers $a$ and $b$.
		\end{enumerate}
		\item For $x\in G$, show that $x$ and $x\inv$ have the same order.
		\item Let $G$ be a finite group and let $x\in G$ be of order $n$.  Prove that if $n$ is odd, then $x = (x^2)^k$ for some $k$.
		\item If $x$ and $g$ are elements of $G$, prove that $|x|=|g\inv xg|$.  Deduce that $|ab| = |ba|$ for all $a,b\in G$.
		\item Suppose $x\in G$ and $|x| = n<\infty$.  If $n=st$ for some positive integers $s$ and $t$, prove that $|x^s| = t$.
		\item If $a$ and $b$ are commuting elements of $G$, prove that $(ab)^n = a^nb^n$ for all $n\in \integers$.
		\item Prove that if $x^2 = 1$ for all $x\in G$, then $G$ is abelian.
		\item Assume $H$ is a nonempty subset of $(G,\star)$ which is closed under $\star$ and under inverses.  Prove that $H$ is a group under $\star$ restricted to $H$ ($H$ is a \emph{subgroup of $G$}).
		\item Prove that if $x$ is an element of the group $G$, then $\left\{ x^n|n\in\integers \right\}$ is a subgroup of $G$ (called the \emph{cyclic subgroup of $G$ generated by $x$}).
		\item Let $(A,\star)$ and $(B,\diamond)$ be groups and let $A\times B$ be their direct procuct.  Verify all the group axioms for $A\times B$.
		\begin{enumerate}
			\item Prove that the associative law holds: for all $(a_i,b_i)\in A\times B, i = 1,2,3$, \\$(a_1,b_1)[(a_2,b_2)(a_3,b_3)] = [(a_1,b_1)(a_2,b_2)](a_3,b_3))$,
			\item Prove that $(1,1)$ is the identity of $A\times B$, and
			\item Prove that the inverse $(a,b)$ is $(a\inv,b\inv)$.
		\end{enumerate}
		\item Prove that $A\times B$ is an abelian group if and only if $A$ and $B$ are.
		\item Prove that the elements $(a,1)$ and $(1,b)$ of $A\times B$ commuted and the order of $(a,b)$ is the least common multiple of $|a|$ and $|b|$.
		\item Prove that any finite group $G$ of even order contains an element of order $2$.
		\item If $x$ is an element of finite order $n$ in $G$, prove that the elements $1$, $x$, $x^2$, ..., $x^{n-1}$ are all distinct.  Deduce that $|x|<G$.
		\item Let $x$ be an element of finite order $n$ in $G$.
		\begin{enumerate}
			\item Prove that if $n$ is odd then $x^i \neq x^{-i}$ for all $i =  1,2,...,n-1$.
			\item Prove that if $n = 2k$ and $1\leq i<n$ then $x^i = x^{-i}$ if and only if $i = k$.
		\end{enumerate}
		\item If $x$ is an element of infinite order in $G$, prove that the elements $x^n$, $n\in\integers$ are all distinct.
		\item If $x$ is an element of finite order in $G$, use the Division Algorithm to show that any integral power of $x$ equals one of the elements in the set of $\left\{ 1,x,x^2,...,x^{n-1} \right\}$.
		\item Assume $G = \left\{ 1,a,b,c \right\}$ is a group of order $4$.  Assume also that $G$ has no element of order $4$.  Use the cancellation laws to show that there is a unique group table for $G$.  Deduce that $G$ is abelian.
	\end{enumerate}
