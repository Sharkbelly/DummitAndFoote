\chapter{Preliminaries}
\section{The Basics}
In $1$ through $4$, let $\mathcal{A}$ be the set of $2\times2$ matrices with real number entries.  Let \[M = \left(
	\begin{array}{cc}
	1& 1 \\ 
	0& 1
	\end{array} 
\right)\]
and let $\mathcal{B} =\left\{X\in \mathcal{A}| MX = XM\right\}$
\begin{enumerate}
	\item Determine which of the following elements of $\mathcal{A}$ lie in $\mathcal{B}$:
		\[
		\left(
		\begin{array}{cc}
		1& 1 \\ 
		0& 1
		\end{array} 
		\right),
		\left(
		\begin{array}{cc}
		1& 1 \\ 
		1& 1
		\end{array} 
		\right)
		\left(
		\begin{array}{cc}
		0& 0 \\ 
		0& 0
		\end{array} 
		\right),
		\left(
		\begin{array}{cc}
		1& 1 \\ 
		1& 0
		\end{array} 
		\right)
		\left(
		\begin{array}{cc}
		1& 0 \\ 
		0& 1
		\end{array} 
		\right),
		\left(
		\begin{array}{cc}
		0& 1 \\ 
		1& 0
		\end{array} 
		\right)
		\]
	\item Prove that if $P,Q\in \mathcal{B}$, then $P+Q\in \mathcal{B}$.
	\item Prove that if $P,Q\in \mathcal{B}$, then $P\cdot Q\in \mathcal{B}$.
	\item Find conditions on $p,q,r,s$ which determine precisely when 
		\[	\left(
		\begin{array}{cc}
		p& q \\ 
		r& s
		\end{array} 
		\right)\in\mathcal{B}\]
	\item Determine whether the following functions $f$ are well defined:
	\begin{enumerate}
		\item $f:\rationals \rightarrow \integers$ defined by $f(a/b) = a$.
		\item $f:\rationals \rightarrow \rationals$ defined by $f(a/b) = a^2/b^2$.
	\end{enumerate}
	\item Determine whether the function $f: \reals\plus \rightarrow\integers$ defined by mapping a real number $r$ to the first digit to the right of the decimal point in a decimal expansion of $r$ is well defined.
	\item Let $f:A\rightarrow B$ be a surjective map of sets.  Prove the relation $a\sim b$ if and only if $f(a) = f(b)$ is an equivalence relation whose equivalence classes are the fibers of $f$.
\end{enumerate}

\section{Properties of the Integers}
\begin{enumerate}
	\item For each of the following pairs of integers $a$ and $b$, determine the GCD, LCM, and write the GCD in the form $ax+by$ for some integers $x$ and $y$.
	\begin{enumerate}
		\item $a=792,b=275$
		\begin{equation*}
			\begin{split}
			792 &= (3)275 + 11 \\
			275 &= (25) 11 + 0\\
			& \Rightarrow GCD(792,275) = 11
			\end{split}
		\end{equation*}
		
		Since $dl = ab$ where $d$ is the GCD of $a$ and $b$ and $l$ is the LCM, we see that $11l = 275\cdot 792$.  Solving for $l$, we find that the LCM of $275$ and $792$ is $19800$.
		
		We can see that $11 = 792(1) - 275(3)$.  
		\item $a=1761,b=1567$
		\begin{equation*}
			\begin{split}
			1761 &= (1)1567 + 194 \\
			1567 &= (8)194 + 15\\
			194 &= (12)15+ 14\\
			15 &= 14+1\\
			& \Rightarrow GCD(1761,1567) = 1
			\end{split}
		\end{equation*}
		
		Since $a$ and $b$ are relatively prime, their LCM is the product $1567\cdot1761 = 2,759,487$.
		
		The GCD can be written as the linear combination $1 = 118(1567) - 105 (1761)$.
	\end{enumerate}
	\item Prove that if the integer $k$ divides the integers $a$ and $b$, then $k$ divides $as+bt$ for every pair of integers $s$ and $t$.
	\begin{proof}
		Let $k,a,b \in \integers$ such that $k|a$ and $k|b$.  This means that $a = km$ and $b = kn$ for some integers $m$ and $n$.  Let $s, t\in \integers$.   Then we see that 
		\begin{equation*}
			\begin{split}
				as+bt &= kms + knt\\
				& = k(ms+nt)
			\end{split}
		\end{equation*}
		Since $k|k(ms+nt)$, the claim is proven.
	\end{proof}
	\item Prove that if $n$ is composite, then there are integers $a$ and $b$ such that $n$ divides $ab$ but $n$ does not divide $a$ nor $b$.
	
	\begin{proof}
		Let $n$ be composite.  Then $n$ may be written as $a\cdot b$ where $1<a,b<n$.  Since $n$ divides itself, $n|a\cdot b$; however, since $a,b<n$, it is impossible for $n$ to divide $a$ or $b$.
	\end{proof}
	
	
	\item Let $a, b$ and $N$ be fixed integers with $a,b\neq 0$ and let $d = (a,b)$.  Suppose $x_0$ and $y_0$ are particular solutions to $ax+by=N$.  Prove that for any integer $t$, the integers \[x = x_0+\frac{b}{d}t \text{ and } y = y_0-\frac{a}{d}t\]
	are also solutions to $ax+by = N$.
	
	\begin{proof}
		Let $a,b,N\in\integers$ with $a,b\neq 0$, and let $d = \gcd(a,b)$.  Let $x_0$ and $y_0$ be solutions to $ax+by = N$, and let $t$ be an arbitrary real number.  
		\begin{align*}
			&a(x_0+\frac{b}{d}t)+b(y_0-\frac{a}{d}t)\\
			=& ax_0+\frac{ab}{d}t + by_0 - \frac{ba}{d}t\\
			=& ax_0+by_0 = N
		\end{align*}
		This proves the claim.
	\end{proof}
	
	\item Determine the value of $\psi(n)$ for each integer $n\leq 30$ where $\psi$ denotes the Euler $\psi$-function.
	
	%
	%
	 %
	 
	 \item Prove the Well Ordering Property of $\integers$ by induction and prove the minimal element is unique.
	 
	 %
	 %
	 %
	 
	 \item If $p$ is prime, prove that there do not exist nonzero integers $a$ and $b$ such that $a^2 = p b^2$ (i.e., $\sqrt{p} \notin \rationals$).
	
	\begin{proof}
		Let $p$ be prime.  Suppose toward a contradiction that $a,b\in\integers$ with $a\neq b$, $a,b\neq 1$ such that $a^2 = pb^2$.  This implies that $$p = \frac{a^2}{b^2}$$
		
		Note that $a = p_1^{\alpha_1}p_{2}^{\alpha_2}...p_{n}^{\alpha_n}$ (by the fundamental theorem of arithmetic).  We can then see that $a^2 = p_1^{2\alpha_1}p_{2}^{2\alpha_2}...p_{n}^{2\alpha_n}$ has an even number of prime factors.  The same can be said of $b$.
		
		Since the qotient $a^2/b^2$ is prime, every prime factor of $b^2$ must be a factor of $a^2$.
		
		There are an even number of such factors, and since $a\neq b$ and $a^2/b^2>1$ there must be a greater (even) number of prime factors of $a^2$.  This implies that the prime factorization of $a^2$ contains at least two factors, so $p$ cannot be prime.  This contradiction proves the claim.
	\end{proof}
	
	 \item Let $p$ be prime, $n\in \integers\plus$.  Find a formula for the largest power of $p$ which divides $n!$.
	 
	 \item Write a computer program to determine $(a,b)$ and to express $(a,b)$ in the form $ax+by$ for some integers $x$ and $y$.
	 \item Prove that for any given positive integer $N$, there exist only finitely many integers $n$ with $\psi(n)= N$.  Conclude in particular that $\psi(n)$ tends to infinity as $n$ tends to infinity.
	 \item Prove that if $d$ divides $n$ then $\psi(d)$ divides $\psi(n)$.
	 
	 \begin{proof}
	 	Let $n,d\in\integers$ such that $d|n$.  We then see that every prime factor $d_i$ of $d$ also divides $n$, and is therefore a prime factor of $n$.  Suppose that $n = p_1^{\alpha_1}p_2^{\alpha_2}...p_m^{\alpha_m}$.  We then know that $$\phi(n) = p_{1}^{\alpha_1-1}(p_1-1)p_{2}^{\alpha_2-1}(p_2-1)...p_{m}^{\alpha_m-1}(p_m-1)$$
	 	Since $d|n$, $d = d_1^{\delta_1}d_2^{\delta_2}...d_n^{\delta_n}$ where $d_i\in\left\{p_i\right\}$.
	 	We also know that
	 	$$\phi(d) = d_{1}^{\delta_1-1}(d_1-1)d_{2}^{\delta_2-1}(d_2-1)...d_{m}^{\delta_m-1}(d_m-1)$$ 
	 	Since every $d_i$ is an element of $\left\{ p_j | 1\leq j \leq m\right\}$, and we may then infer that each $d_i-1$ is an element of $\left\{ p_j-1 | 1\leq j \leq m \right\}$, we see that every factor of $\phi(d)$ is a factor of $\phi(n)$, and the claim is shown.
	 \end{proof}
\end{enumerate}

\section{$\intgp{n}$: The Integers Mod $n$}

\begin{enumerate}
	\item Write all the elements of the residue classes of $\integers/18\integers$.
	\begin{sln}
		\begin{align*}
			& \overline{0} = \left\{ ...,-18,0, 18,... \right\}\\
			& \overline{1} = \left\{  ...,-17, 1, 19,...\right\}\\
			& \overline{2} = \left\{ ..., -16, 2, 20,... \right\}\\
			& \overline{3} = \left\{ ..., -15 ,3 , 21,... \right\}\\
			& \overline{4} = \left\{ ..., -14, 4, 22,... \right\}\\
			& \overline{5} = \left\{ ..., -13, 5, 23,... \right\}\\
			& \overline{6} = \left\{ ..., -12, 6, 24,... \right\}\\
			& \overline{7} = \left\{ ..., -11, 7, 25,... \right\}\\
			& \overline{8} = \left\{ ..., -10, 8, 26,... \right\}\\
			& \overline{9} = \left\{ ..., -9, 9, 27,... \right\}\\
			& \overline{10}=\left\{..., -8, 10, 28,...\right\}\\
			& \overline{11}=\left\{..., -7, 11, 29,...\right\}\\
			& \overline{12}=\left\{..., -6, 12, 30,...\right\}\\
			& \overline{13}=\left\{..., -5, 13, 31,...\right\}\\
			& \overline{14}=\left\{..., -4, 14, 32,...\right\}\\
			& \overline{15}=\left\{..., -3, 15, 33,...\right\}\\
			& \overline{16}=\left\{..., -2, 16, 34,...\right\}\\
			& \overline{17}=\left\{..., -1, 17, 35,...\right\}\\
		\end{align*}
	\end{sln}
	\item Prove that the distinct equivalence classes in $\intgp{n}$ are precisely $\res{0}, \res{1},\res{2}, ..., \res{n-1}$ (use the Division Algorithm).
	
	\item Prove that if $a = a_n10^n+a_{n-1}10^{n-1}+...+a_110+a_0$ is any positive integer, then $a\equiv a_n+a_{n-1}+...+a_1+a_0 \pmod 9$.
	\begin{proof}
		Let $a = a_n10^n+a_{n-1}10^{n-1} + ... + a_1 10^1 + a_0$.  
		We will begin by showing that $$10^n \equiv 1 \pmod 9$$ for any $n\in\integers\plus$ by induction.
		
		Base case: $10^1\equiv 1 \pmod 9$ is clear.
		
		Induction step: Assume that $10^{n}\equiv 1 \pmod 9$.  Then $$10^{n+1} = 10\cdot 10^n \equiv 1\cdot 1 \pmod 9 \equiv 1\pmod 9$$
		We can therefore use the fact that to conclude that $a \equiv a_n + a_{n-1} + ... + a_1 + a_0 \pmod 9$.
	\end{proof}
	\item Compute the remainder when $37^{100}$ is divided by $29$.
	\begin{sln}
		\begin{align*}
			37^{100}& \equiv& 8^{100} \pmod 9\\
			& \equiv& 8^{100} \pmod 9\\
			& \equiv& (2^5)^{60} \pmod 9\\
			& \equiv& 32^{60} \pmod 9\\
			& \equiv& (3^3)^{20} \pmod 9\\
			& \equiv& (-2)^{20} \pmod 9\\
			& \equiv& 2^{20} \pmod 9\\
			& \equiv& (2^5)^4 \pmod 9\\
			& \equiv& 32^4 \pmod 9\\
			& \equiv& 3^3\cdot 3 \pmod 9\\
			& \equiv& -6 \pmod 9\\
			& \equiv& 23 \pmod 9\\
		\end{align*}
	\end{sln}
	\item Compute the last two digits of $9^{1500}$. 
	\begin{sln}
		\begin{align*}
		9^{1500}& = & (9^3)^{500}\\
		& = & 729^{500} \\
		& \equiv& 29^{500} \pmod {100} \\
		& \equiv& 41^{250} \pmod {100} \\
		& \equiv& 81^{500} \pmod {100} \\
		& \equiv& 9^{125}\cdot 9^{125} \pmod {100} \\
		& \equiv& 49^{25}\cdot 49^{25} \pmod {100} \\
		& \equiv& 1^{25} \pmod {100} \\
		& \equiv& 1\pmod {100} \\
		\end{align*}
	\end{sln}
	\item Prove that the squares of the elements in $\integers/4\integers$  are just $\res{0}$ and $\res{1}$.
	\begin{proof}
		Let $\res{a}\in\intgp{4}$.
		\begin{enumerate}
			\item If $\res{a} = \res{0}$
			\item If $\res{a} = \res{2}$
			\item If $\res{a} = \res{0}$
			\item If $\res{a} = \res{0}$
		\end{enumerate}
	\end{proof}
	
	\newpage
	\item Prove for any integers $a$ and $b$ that $a^2+b^2$ never leaves a remainder of $3$ when divided by $4$.
	\begin{proof}
		Let $a,b\in \integers$.  Consider $a^2+b^2$.  Since $a^2\equiv \res{0}$ or $\res{1}\pmod 4$ always, and the same holds for $b^2$, we have $a^2 + b^2 \equiv \res{0}, \res{1},$ or $\res{2}\pmod 4$ always. 
	\end{proof}
	\item Prove that the equation $a^2+b^2 = 3c^2$ has no solutions in nonzero integers $a$, $b$, and $c$.
	\begin{proof}
		By the previous exercise, $a^2+b^2 \equiv \res{0}, \res{1},$ or $\res{2}\pmod 4$ for any integers $a$ and $b$.  By the penultimate exercise, $c^2 \equiv \res{0}$ or $\res{1} \pmod 4$.
		
%	
%	
%	
%	
%	
%	
	\end{proof}
	\item Prove that the square of any odd integer always leaves a rremainder of $1$ when divided by $8$.
	\begin{proof}
		Let $n\in\integers$ be odd.  Then $n = 2\cdot m+1$ for some $m\in\integers$.  We can then consider $n^2$
		\begin{align*}
		n^2 = & (2m+1)^2\\
		 = & 4m^2 + 4m + 1\\
		 = & 4m(m + 1) + 1
		\end{align*}
		If $m$ is even, then $8|4m$; otherwise, $8(m+1)$.  This proves that $4m(m+1) \equiv 0 \pmod 8$, so $n = 4m(m+1) + 1\equiv 1 \pmod 8$.
	\end{proof}
	\item Prove that the number of elements of $(\intgp{n})^\times$ is $\psi(n)$.
	\item Prove that if $\res{a},\res{b}\in(\intgp{n})^\times$, then $\res{a}\cdot\res{b}\in(\intgp{n})^\times$.
	\item Let $n\in\integers$, $n>1$, and let $a\in\integers$ with $1\leq a \leq n$.  Prove if $a$ and $n$ are not relatively prime, there exists an integer $b$ with $1\leq b<n$ such that $ab\equiv 0 \pmod n$ and there cannot be an integer $c$ such that $ac\equiv 1 \pmod n$.
	\begin{proof}
		Let $n\in\integers,n>1$ and $a\in\integers,1\leq a\leq n$.
		Suppose that $\gcd(a,n) \neq 1$.  Then there exists some $m\in\integers\plus$ such that $m|a, m|n$ and $1\leq m\leq a$.  
		Let $b$ be such a number.  
		Then consider $b\cdot a$.  
		Since $b|n$, $b \cdot a \equiv \res{0} \pmod n$.
		
	\end{proof}
	\item Let $n\in\integers$, $n>1$, and let $a\in\integers$ with $1\leq a \leq n$.  Prove that if $a$ and $n$ are relatively prime then there is an integer $c$ such that $ac\equiv 1\pmod n$.
	\item Prove that $(\intgp{n})^\times$ is the set of elements $\res{a}$ of $\intgp{n}$ with $(a,n)=1$. 
	\item For each of the following pairs of integers $a$ and $n$, show that $a$ is relatively prime to $n$ and determine the multiplicative inverse of $\res{a}$ in $\intgp{n}$.
		\begin{enumerate}
			\item $a=13, n=20$
			\item $a=69, n=89$
			\item $a=1891, n=3797$
			\item $a=6003722857, n=77695236973$
		\end{enumerate}
	\item Write a computer program to add and multiply $\pmod n$ for any given $n$.  Outputs should be the least residues of the operations.  Include the feature that if $(a,n)=1$, an integer $c$ between $1$ and $n-1$ such that $\res{a}\res{c} = \res{1}$ may be printed on request.
	
\end{enumerate}